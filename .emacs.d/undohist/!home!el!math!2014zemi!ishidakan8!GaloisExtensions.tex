
((digest . "16eed6f20adbb42b37489fb856fd5160") (undo-list nil (5618 . 5620) (t 21311 39680 477284 612000) nil (5605 . 5607) nil (5602 . 5604) nil (5598 . 5602) ("\\leq" . -5598) ("<
-
" . 5604) ((marker) . -4) ((marker) . -4) ((marker) . -2) ((marker) . -2) (5604 . 5608) (5598 . 5602) nil (5596 . 5597) nil (5595 . 5596) nil (5594 . 5595) nil (5593 . 5595) nil (5592 . 5594) nil (5587 . 5592) nil (5585 . 5586) nil (5584 . 5586) nil (5568 . 5584) nil (5566 . 5567) nil (5565 . 5567) nil (5560 . 5565) nil (5558 . 5560) nil (5556 . 5557) nil (5555 . 5557) nil (5551 . 5555) nil ("b" . -5551) nil ("e" . -5552) nil ("r" . -5553) nil (5550 . 5554) (t 21311 39539 592585 999000) nil undo-tree-canary))
